\documentclass{article}
\usepackage[margin=1in]{geometry}
\usepackage{graphicx}

\usepackage{amsmath}
\usepackage{amssymb}
\usepackage{diffcoeff}

\begin{document}

We consider a simple two reagent reaction diffusion problem

\begin{align}
    \diff{Q}{t} &= \gamma f(Q,R) + \nabla^2 Q\\
    \diff{R}{t} &= \gamma g(Q,R) + D\nabla^2 R
\end{align}

To practice what we've been learning, our goal will be to simulate this system
under progressively more and more complicated circumstances. To begin, we will
not enforce any boundary conditions and inherit the boundary conditions from
the finite element method. The first step of any finite element method is to 
determine the weak form of the problem, and decide how we will update in time.
To keep things simple, let's consider a forward Euler scheme in time which we 
will eventually adapt into a full IMEX scheme.
\begin{align}
    \frac{Q^{n+1}-Q^n}{\Delta t} &= \gamma f(Q^n, R^n) + \nabla^2 Q^n\\
    \frac{R^{n+1}-R^n}{\Delta t} &= \gamma g(Q^n, R^n) + D\nabla^2 R^n
\end{align}
We can easily update the diffusion terms implicitly to help with the stability
of the method. Therefore, we consider the following expression with $n$ shifted
down by one:
\begin{align}
    \frac{Q^{n+1}-Q^{n}}{\Delta t} - \nabla^2 Q^{n+1}  &= \gamma f(Q^{n}, R^{n})\\
    \frac{R^{n+1}-R^{n}}{\Delta t} - D\nabla^2 R^{n+1} &= \gamma g(Q^{n}, R^{n})
\end{align}
Now let's project this into the trial space:
\begin{align}
    (\phi_i^{n+1}, Q^{n+1}) - \Delta t (\phi_i^{n+1}, \nabla^2 Q^{n+1})
    &= (\phi_i^n, Q^{n}) + \gamma \Delta t (\phi_i^{n},\, f(Q^{n}, R^{n}))\\
    (\phi_i^{n+1}, R^{n+1}) - D\Delta t\, (\phi_i^{n+1},\nabla^2 R^{n+1})
    &= (\phi_i^n, R^{n}) + \gamma \Delta t\, (\phi_i^{n}\, g(Q^{n}, R^{n}))
\end{align}
If $Q$ and $R$ are approximately equal to their projections in the trial space,
then
\begin{align}
    (\phi_i^{n+1}, q_j^{n+1}\phi_j^{n+1}) - \Delta t (\phi_i^{n+1}, \nabla^2 q_j^{n+1} \phi_j^{n+1})
    &= (\phi_i^{n}, q_j^{n}\phi_j^{n}) + \gamma \Delta t (\phi_i^{n},\, f(Q^{n}, R^{n}))\\
    (\phi_i^{n+1}, r_j^{n+1} \phi_j^{n+1}) - D\Delta t\, (\phi_i^{n+1},\nabla^2 r_j^{n+1}\phi_j^{n+1})
    &= (\phi_i^n, r_j^{n}\phi_j^{n}) + \gamma \Delta t\, (\phi_i^{n},\, g(Q^{n}, R^{n}))
\end{align}
We will assume homogeneous Dirichlet conditions so that after integrating by
parts the system reduces to
\begin{align}
    (\phi_i^{n+1}, q_j^{n+1}\phi_j^{n+1}) + \Delta t (\nabla \phi_i^{n+1}, \nabla q_j^{n+1} \phi_j^{n+1})
    &= (\phi_i^{n}, q_j^{n}\phi_j^{n}) + \gamma \Delta t (\phi_i^{n},\, f(Q^{n}, R^{n}))\\
    (\phi_i^{n+1}, r_j^{n+1} \phi_j^{n+1}) + D\Delta t\, (\nabla \phi_i^{n+1},\nabla r_j^{n+1}\phi_j^{n+1})
    &= (\phi_i^n, r_j^{n}\phi_j^{n}) + \gamma \Delta t\, (\phi_i^{n},\, g(Q^{n}, R^{n}))
\end{align}
or
\begin{align}
    Mq^n + \Delta t Aq^n
    &= Mq^{n} + \gamma \Delta t f(Q^{n}, R^{n})\\
    Mr^n + D\Delta t Ar^n
    &= Mr^{n} + \gamma \Delta t\, g(Q^{n}, R^{n})
\end{align}
which becomes
\begin{align}
    (M + \Delta t A)q^n
    &= Mq^{n} + \gamma \Delta t f(Q^{n}, R^{n})\\
    (M + D\Delta t A)r^n
    &= Mr^{n} + \gamma \Delta t\, g(Q^{n}, R^{n})
\end{align}

\end{document}

